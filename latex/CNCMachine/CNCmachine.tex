\documentclass[a4paper]{article}
\usepackage{graphicx}
\usepackage{float}
\usepackage{url}

\begin{document}

\title{Proposal: Prototype Printed Circuit Board CNC Machine}
\author{Southampton Projects Group / Reuben Carter}
\maketitle

\section{Problem}

One very difficult and frustrating part of building electronics, is printed circuit board manufacture. lead times when using a commercial services
are usually very slow / expensive, and multiple redesigned boards are often required in the prototyping stage of a design/build if a prototype PCB 
is defective. The solution is to manufacture PCB's in house. The standard industrial technique is to use a pattern mask and chemical etching process, usually with
exposure of a masked light sensitive resist, and a ferric chloride etchant bath to remove copper. The problem with using this method is the chemical
hazard, with the ferric chloride, which is extremely corrosive and readily stains, and the photo-resist solution. For prototyping and one time production, 
this method is quite inefficient. Copper clad board must be cleaned, primed with photo-resist. a pattern must be burned into the resist with a UV light source.
Finally the board can be etched with etchant and cleaned. The whole process may take hours for a single board, and requires specialist facilities to deal with 
chemicals and UV light sources.

\section{Idea}

The CNC machine could provide a cleaner, fast manufacture technique suitable for single board (prototype) manufacture in a standard electronics lab, by cutting cad
designs directly to copper board. The idea is to design a three axis motorized stage, which can position a cutting blade to cut traces onto the board. 
The mechanical assembly must be able to hold, and precisely position both the copper board and the cutting element, while moving fast enough to produce 
prototypes in a reasonable time. 

\end{document}